\documentclass[11pt]{article}
\usepackage[a4paper,left=1.5cm,right=1.5cm,top=1.5cm,bottom=1.5cm]{geometry}
\usepackage{fancyhdr}
\usepackage{mleftright}
\usepackage{verbatim}
\renewcommand{\headrulewidth}{1pt}
\fancyhead[C]{\textsc{[LINMA2380] --- Homework 3}}
\fancyhead[L]{23 November 2020}
\fancyhead[R]{Group 02}

\usepackage{tikz}
\usepackage{pgfplots}
\usepackage[T1]{fontenc}
\usepackage[utf8]{inputenc}
\usepackage[english]{babel}
\usepackage{graphicx}
\usepackage{subcaption}
\usepackage{csquotes}
\usepackage{mathtools,amssymb,amsthm}
\usepackage[binary-units=true,separate-uncertainty = true,multi-part-units=single]{siunitx}
\usepackage{float}
\usepackage[linktoc=all]{hyperref}
\hypersetup{breaklinks=true}
\graphicspath{{img/}}
\usepackage{caption}
\usepackage{textcomp}
\usepackage{array}
\usepackage{color}
\usepackage{tabularx,booktabs}
\usepackage{titlesec}
\usepackage{wrapfig}
\pagestyle{fancy}
\usepackage{mathrsfs}
\usepackage{bm}
\usepackage[ruled,linesnumbered]{algorithm2e}

\DeclarePairedDelimiterX{\norm}[1]{\lVert}{\rVert}{#1}

\newcommand\bovermat[2]{%
  \makebox[0pt][l]{$\smash{\overbrace{\phantom{%
    \begin{matrix}#2\end{matrix}}}^{\text{#1}}}$}#2}
    
\newcommand{\imag}{\mathrm{i}\mkern1mu} % Imaginary unit
\newcommand{\abs}[1]{\left\lvert#1\right\lvert}
\newcommand{\kp}{\otimes}
\DeclareMathOperator{\vect}{vec}

\usepackage{listings}
\lstset{
	language=Python,
	numbers=left,
	numberstyle=\tiny\color{gray},
	basicstyle=\rm\small\ttfamily,
	keywordstyle=\bfseries\color{dkred},
	frame=single,
	commentstyle=\color{gray}=small,
	stringstyle=\color{dkgreen},
	%backgroundcolor=\color{gray!10},
	%tabsize=8, % Thank you Papa Torvalds
	%rulecolor=\color{black!30},
	%title=\lstname,
	breaklines=true,
	framextopmargin=2pt,
	framexbottommargin=2pt,
	extendedchars=true,
	inputencoding=utf8,
}

\DeclareMathOperator*{\argmin}{arg\,min}

\DeclareMathOperator{\rank}{rank}
\DeclareMathOperator{\Ker}{Ker}
\DeclareMathOperator{\vect}{vec}
\DeclareMathOperator{\newdiff}{d} % use \dif instead
\newcommand{\dif}{\newdiff\!}
\newcommand{\e}{\mathrm{e}}

\newcommand{\field}{\mathbb{F}} % field
\newcommand{\real}{\mathbb{R}} % real numbers
\newcommand{\complex}{\mathbb{C}} % complex numbers

\newcommand{\snorm}[1]{\norm{#1}_2} % spectral norm
\newcommand{\fnorm}[1]{\norm{#1}_F} % frobenius norm

\setcounter{MaxMatrixCols}{15}

\newcommand\undermat[2]{% http://tex.stackexchange.com/a/102468/5764
	\makebox[0pt][l]{$\smash{\underbrace{\phantom{%
					\begin{matrix}#2\end{matrix}}}_{\text{$#1$}}}$}#2}

\begin{document}
\section*{Exercise A: Boundedness of trajectories and Lyapunov equation}
\subsection*{A1}
The speed can by derived from the equality y(t) = T x(t):
\begin{align*}
    \Dot{y(t)} &= T \Dot{x(t)} \\
    &= T A x(t) \\
    &= T A T^{-1} y(t) \\
    &= diag\{J_1(\lambda_1), ..., J_r(\lambda_r)\} y(t)
\end{align*}

\subsection*{A2}
We assume that $A=J_n(\lambda)$. Hence we can write :
\begin{equation*}
    x(t)=e^{At}x(0)=e^{J_n(\lambda)t}x(0)
\end{equation*}
We can use the result of question B4 from Homework 1 which develops $e^{J_n(\lambda)}$:
\begin{align*}
    x(t)&=e^{J_n(\lambda)t}x(0)\\
    &=e^{\lambda t}\Bigg(I+\sum_{k=1}^{n-1}\frac{1}{k!}\big(J_n(0)\big)^k\Bigg)^t x(0)\\
    &=e^{\lambda t}e^{J_n(0) t} x(0)\\
    &=e^{\lambda t}\Bigg(I+\sum_{k=1}^{n-1}\frac{1}{k!}\big(J_n(0) t\big)^k\Bigg) x(0)\\
    &=e^{\lambda t}\Bigg(I+\sum_{k=1}^{n-1}\frac{1}{k!}J_n^k(0) t^k\Bigg) x(0)
\end{align*}
The third and fourth equalities come from the definition of the matrix exponential. We can now develop the terms in parentheses : 
\begin{align*}
    I+\sum_{k=1}^{n-1}\frac{1}{k!}J_n^k(0) t^k = I &+ \frac{t}{1!}
    \begin{pmatrix}
    0&1&&\\
    &\ddots&\ddots&&\\
    &&0&1&\\
    &&&0&1\\
    &&&&0
    \end{pmatrix} + \frac{t^2}{2!}
    \begin{pmatrix}
    0&0&1&\\
    &\ddots&\ddots&\ddots&\\
    &&0&0&1\\
    &&&0&0\\
    &&&&0
    \end{pmatrix}\\ \\
    &+ \dots +\frac{t^k}{k!}
    %\begin{pmatrix}
    %\bovermat{k zero columns}{0&\dots&0}&1&\\
    %&\ddots&&\ddots&\ddots&\\
    %&&0&\dots&0&1\\
    %&&&0&\dots&0\\
    %&&&&\ddots&\vdots\\
    %&&&&&0
    %\end{pmatrix}
    \begin{pmatrix}
    \bovermat{$k$ zeros}{0&\dots&0}&1&\\
    &\ddots&&\ddots&\ddots&\\
    &&\ddots&&\ddots&1\\
    &&&\ddots&&0\\
    &&&&\ddots&\vdots\\
    &&&&&0
    \end{pmatrix}
    +\dots+\frac{t^{n-1}}{(n-1)!}
    \begin{pmatrix}
    \bovermat{$n-1$ zeros}{0&\dots&0}&1\\
     &\ddots&&0\\
     & & \ddots & \vdots\\
     & & & 0\\
    \end{pmatrix}\\
    &=
    \begin{pmatrix}
    1 & t/1! & \dots & t/k! & \dots & t^{n-1}/(n-1)!\\
      & \ddots & \ddots &  & \ddots & \vdots\\
      & & 1 & t/1! & & t/k! &\\
      & & & \ddots & \ddots & \vdots\\
      & & & & 1 & t/1!\\
      & & & & & 1\\
    \end{pmatrix}
\end{align*}
We deduce from the previous expression that for $i\in [n]$:
\begin{equation*}
    x_i(t)=e^{\lambda t}\sum^{n}_{j=i}\frac{1}{(j-i)!}t^{j-i}x_j(0)
\end{equation*}
\subsection*{A3}
The minimal polynomial of a matrix \(A \in \complex^{n \times n}\) is the polynomial
\[
m(\lambda) = \prod_i (\lambda - \lambda_i)^{k_i^*},
\]
where
\[
f(J) = \mathop{\mathrm{diag}} \left\{f\left(J_{k_{i_j}}(\lambda_{i_j})\right)\right\}, \quad k_i^* = \max_{1 \leqslant j \leqslant n_i} k_{i_j},
\]
and \(n_i\) is the number of Jordan blocks with eigenvalue \(\lambda_i\).

Simple eigenvalues thus have the property that the size of the largest Jordan block with that eigenvalue \(\lambda_i\) is \(1\) (i.e. \(k_i^* = 1\)).
\subsection*{A4}
=>\\
By A1, we know there exists a change of coordinates $y=Tx$ such that $y(t)$ can be decomposed as follows: $y(t) = [y_1(t)^T, . . . , y_r(t)^T]^T$,where each $y_i(t)$, $i \in [r]$, is the trajectory of a continuous-time linear dynamical system with a Jordan block as transition matrix.\\
By A2, we have an expression for the time course of the components $y_i(t)$.\\
If $\lambda_i$ is such that $Re(\lambda_i)<0$, then from A2 we clearly see that the expression $y_i(t)$ tends to zero. If $\lambda_i$ is such that $Re(\lambda_i)<0$, by A3 we know the size of the largest Jordan block with that eigenvalue is 1, hence the expression of $y_i(t)$ simplifies to $y_i(t)=e^{\lambda t} y_j(0)$ which tends to zero when t tends to infinity.\\
We've shown $y_i(t)$ and hence $\norm{y(t)}=\norm{Tx(t)}$ tends to zero.
\subsection*{A5}
% not really sure this is what they're asking for
\begin{proof}
	One can show that the matrix \(P\) must exist by proving that for \(P = I_n\), the statement holds.
	Indeed, \(I_n\) is positive definite and Hermitian.
	Since \(D\) and \(D^*\) are diagonal, their sum is also diagonal.
	In fact,
	\[
	D + D^* = \mathop{\mathrm{diag}}(\lambda_1 + \lambda_1^*, \dots, \lambda_n + \lambda_n^*),
	\]
	where \(\lambda_i\) and \(\lambda_i^*\) are the eigenvalues of \(D\) and \(D^*\), respectively.
	We know that the imaginary parts of these eigenvalues cancel each other out, and that the real parts are the same (and are nonpositive), by virtue of the definition of the conjugate transpose.
	We thus get
	\[
	D^*P + PD = D^*I_n + I_nD = D^* + D = \mathop{\mathrm{diag}}(\lambda_1 + \lambda_1^*, \dots, \lambda_n + \lambda_n^*) \preceq 0,
	\]
	which proves the existence of such a positive definite Hermitian matrix \(P\).
\end{proof}
\subsection*{A6}
First, we develop $B=I\kp A^*+A^T\kp I$:
\begin{align*}
    B=
    \begin{pmatrix}
    A^* & & \\
    &\ddots&\\
    & & A^*
    \end{pmatrix}+
    \begin{pmatrix}
    a_{11} I & \dots & a_{1n} I\\
    \vdots & & \vdots\\
    a_{n1} I & \dots & a_{nn} I\\
    \end{pmatrix}
\end{align*}
We know $A\in\complex^{n\times n}$ is a Jordan block with eigenvalue $\lambda$:
\begin{align*}
    B&=
    \renewcommand{\arraystretch}{1.5}
    \begin{pmatrix}
    J_n(\lambda)^* & & \\
    &\ddots&\\
    & & J_n(\lambda)^*
    \end{pmatrix}+
    \begin{pmatrix}
    \lambda I & & &\\
    I & \ddots & &\\
    & \ddots & \ddots &\\
    & & I & \lambda I
    \end{pmatrix}\\
    &=\begin{pmatrix}
    J_n(\lambda)^*+\lambda I &  &  & \\
    I & \ddots & &\\
    & \ddots & \ddots &\\
    & & I & J_n(\lambda)^*+\lambda I
    \end{pmatrix}\\
    &=\begin{pmatrix}
    J_n(\lambda+\lambda^*)^* & & & \\
    I & \ddots & &\\
    & \ddots & \ddots &\\
    & & I & J_n(\lambda+\lambda^*)^*
    \end{pmatrix}
\end{align*}
The matrix B is a lower triangular and therefore its eigenvalues are the diagonal elements : $\lambda+\lambda^*$. Moreover, we know $Re(\lambda)>0$ and so $\lambda+\lambda^*=2Re(\lambda)>0$. From this we deduce that $B$ is positive definite and hence invertible. It follows that the system $B\vect(P)=-\vect(Q)$ has a unique solution and consequently we conclude that P exists and is unique.\\
Next we show that if $P\in\complex^{n\times n}$ satisfies $A^*P+PA=-Q$ then $P^*$ also satisfies $A^*P^*+P^*A=-Q$. This can be simply proven by taking the transpose conjuguate of both sides of the equation $A^*P+PA=-Q$ that P satisfies taking into account that Q is hermitian:
\begin{align*}
    (A^*P+PA)^*&=(-Q)^*\\
    P^*A+A^*P^*&=-Q
\end{align*}
This shows that $P^*$ satisfies $A^*P^*+P^*A=-Q$.\\
Combining the two last results, we deduce that there always exists a unique hermitian matrix $P$ satisfying $B\vect(P)=-\vect(Q)$.\\
Finally, we want to show that $P$ is hermitian. We consider a trajectory $x(t)$ starting from $x(0)$. If we define $V(x(t))=x(t)^*Px(t)$, we observe that:
\begin{align*}
    \dot{V}(x(t))&=\dot{x}(t)^*P x(t)+x(t)^*P\dot{x}(t)\\
    &=x(t)^*A^*P x(t)+x(t)^*P A x(t)\\
    &=x(t)^*(A^*P+P A) x(t)\\
    &=x(t)^*(-Q) x(t)\\
    &\leq 0
\end{align*}
Consequently, we find that for $t>0$ we have: $x(t)^*P x(t)< x(0)^*P x(0)$.
%%%% TROUVER LA DERNIERE PARTIE
\subsection*{A7}
2 => 3\\
%IDEE
% prouver pourquoi on peut considerer la jordan form
Jordan form of A such that simple eigenvalues in block $A_1$ and other eigenvalues in block $A_2$:
\begin{align*}
    \begin{pmatrix}
    A_1 & \\
    & A_2
    \end{pmatrix}
\end{align*}
By A3 we find $P_1$ such that $P_1A_1+A_1^*P_1=-Q1$ and by A6 we find $P_2$ such that $P_2A_2+A_2^*P_2=-Q2$
\begin{align*}
    \begin{pmatrix}
    P_1 & \\
    & P_2
    \end{pmatrix}
    \begin{pmatrix}
    A_1 & \\
    & A_2
    \end{pmatrix}
    +
    \begin{pmatrix}
    A_1^* & \\
    & A_2^*
    \end{pmatrix}
    \begin{pmatrix}
    P_1 & \\
    & P_2
    \end{pmatrix}=-
    \begin{pmatrix}
    Q_1 & \\
    & Q_2
    \end{pmatrix}\preceq 0
\end{align*}

\subsection*{A8}

\section*{Exercise B: Implementation}
\subsection*{B1}
First, $I \otimes A^{*} + A^{T} \otimes I = V (I \otimes S^{*} + S^{T} \otimes I) V^{*}$ has to be demonstrated:\\

\begin{align*}
   V (I \otimes S^{*} + S^{T} \otimes I) V^{*} &= (U^{*} \otimes U) (I \otimes S^{*} + S^{T} \otimes I) (U^{*} \otimes U)^{*} \\
   &= [(U^{*} \otimes U) (I \otimes S^{*}) + (U^{*} \otimes U) (S^{T} \otimes I)] (U \otimes U^{*}) \\
   &= [(U^{*} I \otimes U S^{*}) + (U^{*}S^{T} \otimes U I)] (U \otimes U^{*}) \\
   &= (U^{*} I \otimes U S^{*}) (U \otimes U^{*}) + (U^{*}S^{T} \otimes U I) (U \otimes U^{*}) \\
   &= (U^{*} I U \otimes U S^{*} U^{*}) + (U^{*}S^{T}U \otimes U I U^{*}) \\
   &= (I \otimes U (U S)^{*} ) + (U^{*} (U^{T} S)^{T} \otimes I) \\
   &= (I \otimes (U S U^{*})^{*} ) + ((U^{T} S U^{*T})^{T} \otimes I) \\
   &= (I \otimes A^{*} ) + (A^{T} \otimes I) \\
\end{align*}

Secondly, the demonstration of V being unitary:\\

\begin{align*}
   V V^{*} &= I \\
   &= (U^{*} \otimes U) (U^{*} \otimes U)^{*} \\
   &= (U^{*} \otimes U) (U \otimes U^{*}) \\
   &= U^{*} U \otimes U U^{*} \\
   &= I \otimes I \\
   &= I
\end{align*}

Thirdly, $I \otimes S^{*} + S^{T} \otimes I$ is lower triangular because S is upper triangular, thus $S^{T}$ and $S^{*}$ and lower triangular. Their Kronecker products with the identity matrix remain lower triangular matrices and finally their sum remains a lower triangular matrix. \\

Before presenting the method to solve $(I \otimes A^{*} + A^{T} \otimes I) vec(P) = - vec(Q)$, let's introduce a lemme \label{lemme}
Consider two matrices C,D partitioned in blocks of size $n_{1} x n_{p}, n_{2} x n_{p}, ..., n_{p-1} x n_{p}$ according to
\begin{align*}
P = \begin{pmatrix} P_{1} \\ \vdots \\ P_{p-1} \end{pmatrix} \in R^{N x n_{r}}, D = \begin{pmatrix} D_{1} \\ \vdots \\ D_{p-1} \end{pmatrix} \in R^{N x n_{p}}
\end{align*}
where $C_{}, D_{j} \in R^{n_{j}xn_{p}}$ and $N = \sum_{j=1}^{p-1} n_{j}$. Let $T \in R^{nxn}$ be a block triangular matrix partitioned as P and D. For any $R_22 \in R^{n_{p}xn_{p}}$, if P satisfies the equation $D = T P + P R_{22}^{T}$, then $P_{j}$, p-1, p-2, ..., 1 satisfy 
\begin{align}
    T_{jj} P_{j} + P_{j} R_{22}^{T} = D_{j}^{\sim}
\end{align} \label{lemme}
where $D_{j}^{\sim} = D_{j} - \sum_{i=j+1}^{p-1} T_{ji} P_{i}$.

The method to solve $(I \otimes A^{*} + A^{T} \otimes I) vec(P) = - vec(Q)$ is the following:\\
Since all matrices have a Schur decomposition: there exists matrices U and S such that:$A = U S U^{*}$ where U is an orthogonal matrix and $S \in R^{nxn}$ a block-triangular matrix where
\begin{align}
S = \begin{pmatrix} S_{11}  & \hdots & T_{1,r} \\  & \ddots & \vdots \\ & & T_{r,r}
\end{pmatrix}
\end{align} \label{matrix_S}
and $S_{jj} \in R^{n_{j}xn_{j}}$, $n_{j} \in \{1, 2\}$, j = 1, ..., r and $\sum_{j=1}^{r} n_{j} = n$. \\
The Lyapunov equation is multiplied from the right and left with U and $U^{*}$ respectively,
\begin{align}
- U Q U^{*} &= U A^{*} P U^{*} + U P A U^{*} \\
&= U A^{*} U^{*} U P U^{*} + U P U^{*} U A U^{*} \\
&= S Y + Y S^{*} \label{syys}
\end{align}
where $Y = U^{*} P U^{*}$. \\
Matrices and corresponding blocks are introduced such that
\begin{align}
    - U Q U^{*} = \begin{pmatrix} C_{11} & C_{12} \\ C_{21} & C_{22} \end{pmatrix}, Y = \begin{pmatrix} Z_{11} & Z_{12} \\ Z_{21} & Z{22} \end{pmatrix}, S = \begin{pmatrix} R_{11} & R_{12} \\ 0 & R_{22} \end{pmatrix}
\end{align}
where the blocks are such that $Z_{22}, C_{22}, S_{rr} = R_{22} \in R^{n_{r}xn_{r}}$ (the size of the last block of S). \\
This triangularized problem can be solved with backward substitution. \\
The following equations are obtained by separating thr blocks in the equation (\ref{syys})
\begin{align}
    C_{11} &= R_{11} Z_{11} + R_{12} Z_{21} + Z_{11} R_{11}^{T} + Z_{12} R_{12}^{T} \label{C11} \\
    C_{12} &= R_{11} Z_{12} + R_{12} Z_{22} + Z_{12} R_{22}^{T} \\
    C_{21} &= R_{22} Z_{21} + Z_{21} R_{11}^{T} + Z_{22} R_{12}^{T} \\
    C_{22} &= R_{22} Z_{22} + Z_{22} R_{22}^{T} \label{C22}
\end{align}

The last equation of size $n_{r} x n_{r}$ can be solved explicitly since $n_{r} \in \{1, 2\}$ thanks to the choice of block sizes:\\
if $n_{r} = 1$, $Z_{22}$ is scalar: $Z_{22} = \frac{C_{22}}{2 R_{22}}$ \label{Y22} \\
if $n_{r} = 2$, the equation \ref{C22} can be solved easily because it is a 2x2 Lyapunov equation. It can be solved with:$vec(Z_{22} = (I \otimes R_{22} + R_{22} \otimes I)^{-1} vec(C_{22})$ \label{vecY22}.\\
Thanks to the lemme \ref{lemme}, the now known matrix $Z_{22}$ can be insert in the others:\\
\begin{align}
C_{12}^{\sim} &= C_{12} - R_{12} Z_{22} = R_{11} Z_{12} + Z_{12} R_{22}^{T} \label{C12sim} \\ 
C_{21}^{\sim} &= C_{21}^{T} - R_{12} Z_{21}^{T} + Z_{21}^{T} R_{22}^{T} \label{C21sim}
\end{align}
$P_{j}$ can be computed explicitly from a small linear system $vec(P_j) = (I \otimes S_{jj} + R_{22} \otimes I)^{-1} vec(-Q_{j}^{\sim})$ \label{vecpj}. By solving \ref{vecpj} for j= p-1, ..., 1 for both equations of (\ref{C12sim}) and (\ref{C21}), solutions for $Z_{12}$ and $Z_{21}$ are obtained. Insertion of $Z_{12}, Z_{21}$ and $Z_{22}$ into (\ref{C11}) gives a new Lyapunov equation of size $n-n_{p}$ and the process can be repeated for the smaller matrix.

\begin{algorithm}[H]
Compute the real Schur decomposition [U, S] = schur(A) and establish $n_{1}, ..., n_{r}, S_{12}, S_{1r}, ..., S_{rr}$ with partitioning according to \ref{matrix_S}\\
Set $C = U (-Q) U^{*}$ \\
Set m = n \\
\For{k = r, ..., 1 do}
{Set $m = m - n_{k}$ \\
Partition the matrix C with $C_{11}, C_{12}, C_{21}, C_{22}$ according to C = \begin{pmatrix}
C_{11} & C_{12} \\ C_{21} & C_{22} \end{pmatrix} with $C_{22} \in R^{n_{k}xn_{k}}$ \\
Set $R_{22} = S_{kk}$ and $R_{11}$ = \begin{pmatrix} S_{11} & \hdots & S_{1,k-1} \\ & \ddots & \vdots \\ & & S_{k-1,k-1} \end{pmatrix}, $R_{12}$ = \begin{pmatrix} S_{1,k} \\ \vdots \\ S_{k-1,k} \end{pmatrix} \\
Solve \ref{C22} for $Z_{22} \in R^{n_{k}xn_{k}}$ using \ref{Y22} or \ref{vecY22} \\
Compute $C_{12}^{\sim}, C_{21}^{\sim}$ using \ref{C12sim} and \ref{C21sim} \\
Solve \ref{C12sim} and \ref{C21sim} for $Z_{12} \in R^{mxn_{k}}$ and $Z_{21} \in R^{n_{k}xm}$ using the lemme \ref{lemme} and \ref{vecpj} with p=j \\
Store Y(1:m,m+(1:$n_{k}$))=$Z_{12}$ \\
Store Y(m(1:$n_{k}$), 1:m) = $Z_{21}$ \\
Store Y(m+(1:$n_{k}$), m+(1:$n_{k}$))=$Z_{22}$ \\
Set $C = C_{11} - R_{12} Z_{21} - Z_{12} R_{12}^{T}$} \\
end
\Return{solution $P = U Y U^{*}$}
\caption{ }
\end{algorithm}


\subsection*{B2}
\end{document}