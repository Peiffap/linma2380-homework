\documentclass[11pt]{article}
\usepackage[a4paper,left=1.5cm,right=1.5cm,top=1.5cm,bottom=1.5cm]{geometry}
\usepackage{fancyhdr}
\renewcommand{\headrulewidth}{1pt}
\fancyhead[C]{\textsc{[LINMA2380] --- Homework 1}}
\fancyhead[L]{12 October 2020}
\fancyhead[R]{Group 02}

\usepackage{tikz}
\usepackage{pgfplots}
\usepackage[T1]{fontenc}
\usepackage[utf8]{inputenc}
\usepackage[english]{babel}
\usepackage{graphicx}
\usepackage{subcaption}
\usepackage{csquotes}
\usepackage{mathtools,amssymb,amsthm}
\usepackage[binary-units=true,separate-uncertainty = true,multi-part-units=single]{siunitx}
\usepackage{float}
\usepackage[linktoc=all]{hyperref}
\hypersetup{breaklinks=true}
\graphicspath{{img/}}
\usepackage{caption}
\usepackage{textcomp}
\usepackage{array}
\usepackage{color}
\usepackage{tabularx,booktabs}
\usepackage{titlesec}
\usepackage{wrapfig}
\pagestyle{fancy}
\usepackage{mathrsfs}
\usepackage{bm}
\DeclarePairedDelimiterX{\norm}[1]{\lVert}{\rVert}{#1}

\newcommand{\imag}{\mathrm{i}\mkern1mu} % Imaginary unit
\newcommand{\abs}[1]{\left\lvert#1\right\lvert}
\usepackage{listings}
\lstset{
	language=Python,
	numbers=left,
	numberstyle=\tiny\color{gray},
	basicstyle=\rm\small\ttfamily,
	keywordstyle=\bfseries\color{dkred},
	frame=single,
	commentstyle=\color{gray}=small,
	stringstyle=\color{dkgreen},
	%backgroundcolor=\color{gray!10},
	%tabsize=8, % Thank you Papa Torvalds
	%rulecolor=\color{black!30},
	%title=\lstname,
	breaklines=true,
	framextopmargin=2pt,
	framexbottommargin=2pt,
	extendedchars=true,
	inputencoding=utf8,
}

\DeclareMathOperator{\newdiff}{d} % use \dif instead
\newcommand{\dif}{\newdiff\!}
\newcommand{\e}{\mathrm{e}}

\newcommand{\field}{\mathbb{F}} % field
\newcommand{\kp}{\otimes} % kronecker product

% TODO
% - copy the statements (?)

\begin{document}
\section*{Exercise A: The Kronecker product}
\subsection*{A1}
The Kronecker product of two matrices \(A \in \field^{m \times n}\) and \(B \in \field^{p \times q}\) is the matrix of size \(mp \times nq\) whose elements are all possible products between the elements of \(A\) and \(B\) arranged in the following way:
\[
A \kp B \coloneqq \begin{bmatrix}
a_{11} B & \cdots & a_{1n}B \\
\vdots & \ddots & \vdots \\
a_{m1} B & \cdots & a_{mn} B
\end{bmatrix}.
\]

\subsection*{A2}
The Kronecker product is associative.
Let \(C \in \field^{s \times t}\) be a third matrix.
We show that \((A \kp B) \kp C = A \kp (B \kp C)\).
\begin{proof}
\begin{align*}
(A \kp B) \kp C &= \begin{bmatrix}
a_{11} B & \cdots & a_{1n}B \\
\vdots & \ddots & \vdots \\
a_{m1} B & \cdots & a_{mn} B
\end{bmatrix} \kp C \\
&= \begin{bmatrix}
a_{11} b_{11} C & \cdots & a_{11} b_{1q} C & \cdots & a_{1n} b_{11} C & \cdots & a_{1n} b_{1q} C \\
\vdots & \ddots & \vdots & \ddots & \vdots & \ddots & \vdots \\
a_{11} b_{p1} C & \cdots & a_{11} b_{pq} C & \cdots & a_{1n} b_{p1} C & \cdots & a_{1n} b_{pq} C \\
\vdots & \ddots & \vdots & \ddots & \vdots & \ddots & \vdots \\
a_{m1} b_{11} C & \cdots & a_{m1} b_{1q} C & \cdots & a_{mn} b_{11} C & \cdots & a_{mn} b_{1q} C \\
\vdots & \ddots & \vdots & \ddots & \vdots & \ddots & \vdots \\
a_{m1} b_{p1} C & \cdots & a_{m1} b_{pq} C & \cdots & a_{mn} b_{p1} C & \cdots & a_{mn} b_{pq} C
\end{bmatrix}\\
&= A \kp (B \kp C).\qedhere
\end{align*}
\end{proof}

The Kronecker is non-commutative; we show that \(A \kp B \neq B \kp A\)

\begin{proof}
We show a counterexample to the claim of commutativity.
Let
\[
A = \begin{bmatrix}
2 & 3 \\ 0 & 1
\end{bmatrix}, \quad, B = \begin{bmatrix}
0 & -1 \\ -1 & 1
\end{bmatrix}.
\]
In that case, we have
\[
A \kp B = \begin{bmatrix}
0 & -2 & 0 & -3 \\
-2 & 2 & -3 & 3 \\
0 & 0 & 0 & -1 \\
0 & 0 & -1 & 1
\end{bmatrix}, \quad B \kp A = \begin{bmatrix}
0 & 0 & -2 & -3 \\
0 & 0 & 0 & -1 \\
-2 & -3 & 2 & 3 \\
0 & -1 & 0 & 1
\end{bmatrix}.
\]
We see that \(A \kp B \neq B \kp A\).
\end{proof}

Finally, the set \(\field^{n \times n}\) equipped with the Kronecker product is a group by virtue of it being a field. % TODO check this

\subsection*{A3}
Let \(A \in \field^{m \times n}\), \(B \in \field^{p \times q}\), \(C \in \field^{n \times r}\), and \(D \in \field^{q \times s}\)
\begin{proof}
We simply verify that
\begin{align*}
(A \kp B) (C \kp D) &= \begin{bmatrix}
a_{11} B & \cdots & a_{1n}B \\
\vdots & \ddots & \vdots \\
a_{m1} B & \cdots & a_{mn} B
\end{bmatrix} \begin{bmatrix}
c_{11} D & \cdots & c_{1r} D \\
\vdots & \ddots & \vdots \\
c_{n1} D & \cdots & c_{nr} D
\end{bmatrix}\\
&= \begin{bmatrix}
\sum_{k=1}^n a_{1k} c_{k1} BD & \cdots & \sum_{k=1}^n a_{1k} c_{kr} BD \\
\vdots & \ddots & \vdots \\
\sum_{k=1}^n a_{mk} c_{k1} BD & \cdots & \sum_{k=1}^n a_{mk} c_{kr} BD
\end{bmatrix}\\
&= AC \kp BD.
\end{align*}
\end{proof}

This allows us to say that (if \(A \in \field^{n \times n}\) and \(B \in \field^{m \times m}\) are nonsingular)
\[
(A \kp B)(A^{-1} \kp B^{-1}) = AA^{-1} \kp BB^{-1} = I_n \kp I_m = I_{nm},
\]
and hence that
\[
(A \kp B)^{-1} = A^{-1} \kp B^{-1}.\qedhere
\]

\subsection*{A4}
We first show the first property, P1.
\begin{proof}
By induction.
The base case is trivial:
\[
A^{\kp 1} B^{\kp 1} = AB = (AB)^{\kp 1}.
\]
Next, we assume the property holds for \(k = n\), and we prove it for \(k = n+1\):
\begin{align*}
A^{\kp k+1} B^{\kp k+1} &= (A^{\kp k} \kp A) (B^{\kp k} \kp B)\\
&\overset{\textnormal{A3}}{=}(A^{\kp k} B^{\kp k}) \kp AB\\
&= (AB)^{\kp k} \kp AB\\
&= (AB)^{\kp k+1}.\qedhere
\end{align*}
\end{proof}

Next, we show the second property, P2.
\begin{proof}
We start by proving an auxiliary lemma, L1.
\[
(A \kp B)^\top = \begin{bmatrix}
a_{11} B & \cdots & a_{1n}B \\
\vdots & \ddots & \vdots \\
a_{m1} B & \cdots & a_{mn} B
\end{bmatrix}^\top = \begin{bmatrix}
a_{11} B^\top & \cdots & a_{m1}B^\top \\
\vdots & \ddots & \vdots \\
a_{1n} B^\top & \cdots & a_{mn} B^\top
\end{bmatrix} = A^\top \kp B^\top.
\]

We then proceed by induction.
The base case is trivial as before:
\[
(A^{\kp 1})^\top = A^\top = (A^\top)^{\kp 1}.
\]
Next, we assume the property holds for \(k = n\), and we prove it for \(k = n+1\):
\begin{align*}
(A^{\kp k+1})^\top &= (A^{\kp k} \kp A)^\top\\
&\overset{\textnormal{L1}}{=} (A^{\kp k})^\top \kp A^\top\\
&= (A^\top)^{\kp k} \kp A^\top\\
&= (A^\top)^{\kp k+1}.\qedhere
\end{align*}
\end{proof}

Finally, we show the following:
\[
\norm{v^{\kp k}} = \norm{v}^k.
\]
\begin{proof}
\begin{align*}
\norm{v^{\kp k}} &= \sqrt{(v^{\kp k})^\top v^{\kp k}}\\
&\overset{\textnormal{P2}}{=} \sqrt{(v^\top)^{\kp k} v^{\kp k}}\\
&\overset{\textnormal{P1}}{=} \sqrt{(v^\top v)^{\kp k}}\\
&= \sqrt{(v^\top v)^k}\\
&= \Big(\sqrt{v^\top v}\,\Big)^k\\
&= \norm{v}^k,
\end{align*}
where the fourth equality follows from a simplification of the Kronecker product for scalars, and the fifth equality is a property of the square root.
\end{proof}

\subsection*{A5}
The determinant of a square matrix \(A \in \field^{n \times n}\) as
\[
\det(A) = \sum_{\bm{j}}(-1)^{t(\bm{j})} a_{1 j_1} \cdot a_{2 j_2} \dots a_{n j_n},
\]
where the index vector \(\bm{j}\) constitutes a permutation of \(\{1, 2, \dots, n\}\), and \(t(\bm{j})\) denotes the parity of each quasi-diagonal.

Next, we show that \(\det(A \kp I_m) = \det(A)^m\).
\begin{proof}
\begin{align*}
\det(A \kp I_m) &= \det \left(\begin{bmatrix}
a_{11} I_m & \cdots & a_{1n} I_m \\
\vdots & \ddots & \vdots \\
a_{n1} I_m & \cdots & a_{nn} I_m
\end{bmatrix}\right)\\
&=\det \left(\begin{bmatrix}
a_{11} & \cdots & 0 & \cdots & a_{1n} & \cdots & 0\\
\vdots & \ddots & \vdots & \vdots & \ddots & \vdots \\
0 & \cdots & a_{11} & \cdots & 0 & \cdots & a_{1n}\\
\vdots &  & \vdots &  & \vdots & \vdots \\
a_{n1} & \cdots & 0 & \cdots & a_{nn} & \cdots & 0\\
\vdots & \ddots & \vdots & \vdots & \ddots & \vdots \\
0 & \cdots & a_{n1} & \cdots & 0 & \cdots & a_{nn}\\
\end{bmatrix}\right)
\end{align*}
The Laplace theorem states that for a matrix B of dimensions \(n\times n\) and a p-tuple of rows $\bm{i}_p$, we have :
\begin{align*}
\det(B) &= \sum_{\bm{j}_p} B {\bm{i}_p\choose \bm{j}_p} B^c {\bm{i}_p\choose \bm{j}_p}
\end{align*}
We apply this theorem with $B=A \kp I_m$ with the n-tuple $\bm{i}_n$ : $(1,m+1,2m+1,...,(n-1)m+1)$. For every n-tuple $\bm{j}_n$ that contains another index than these present in $\bm{i}_n$, the minor $B {\bm{i}_n\choose \bm{j}_n}$ is zero. Indeed, if we consider only the rows whose indices are in $\bm{i}_n$, we have :
\begin{align*}
B'&=\begin{pmatrix}
a_{11} & 0 & \cdots & 0 & a_{12} & 0 & \cdots & 0 & a_{1n} & 0 & \cdots & 0\\
a_{21} & 0 & \cdots & 0 & a_{22} & 0 & \cdots & 0 & a_{2n} & 0 & \cdots & 0\\
\vdots & \vdots & & \vdots & \vdots & \vdots & & \vdots & \vdots & \vdots & & \vdots \\
a_{n1} & 0 & \cdots & 0 & a_{n2} & 0 & \cdots & 0 & a_{nn} & 0 & \cdots & 0 \\
\end{pmatrix}
\end{align*}

Thus if $\bm{j}_n$ contains any index not present in $\bm{i}_n$, a column full of zero is included leading to a null determinant. We are left with :
\begin{align}
\det(A \kp I_m) &= B {\bm{i}_n\choose \bm{i}_n} B^c {\bm{i}_n\choose \bm{i}_n}\\ \label{recc}
&= \det(A) \det(A \kp I_{m-1})
\end{align}
The first term of the second line can be easily found by inspecting B'.
%% add details
From \ref{recc}, we conclude that \(\det(A \kp I_m) = \det(A)^m\).
\end{proof}

\subsection*{A6}
The rank of a matrix \(A \in \field^{m \times n}\) is equal to the largest size of its nonzero minors.

Next, we prove the property : $\rank(A\kp B)=\rank(A)\rank(B)=\rank(B\kp A)$.

\begin{proof}
First, we note that $B\kp A$ can be obtained by inverting rows and columns of $A\kp B$. As elementary operations do not affect the rank of a matrix, we deduce that $\rank(A\kp B)$ is equal to $\rank(B\kp A)$.
Let $R_1$ and $Q_1$ be products of elementary transformations such that :
\begin{align*}
    R_1BQ_1 &=\begin{bmatrix}
    I_r & 0_{r\times(q-r)}\\
    0_{(p-r)\times r} & 0_{(p-r)\times (q-r)}
    \end{bmatrix}
\end{align*}
\begin{align*}
    \begin{pmatrix}
    R_1 & &\\
    & \ddots & \\
    & & R_1
    \end{pmatrix}
    \begin{pmatrix}
    a_{11}B & \cdots & a_{1n}B\\
    \vdots & & \vdots\\
    a_{m1}B & \cdots & a_{mn}B
    \end{pmatrix}
    \begin{pmatrix}
    Q_1 & &\\
    & \ddots & \\
    & & Q_1
    \end{pmatrix}
    &= \begin{pmatrix}
    a_{11}R_1BQ_1 & \cdots & a_{1n}R_1BQ_1\\
    \vdots & & \vdots\\
    a_{m1}R_1BQ_1 & \cdots & a_{mn}R_1BQ_1
    \end{pmatrix}\\
%    &= \begin{pmatrix}
%    a_{11}I_r & 0_{r\times(q-r)} & \cdots & a_{1n}I_r & 0_{r\times(q-r)}\\
%    0_{(p-r)\times r} & 0_{(p-r)\times (q-r)} & \cdots & 0_{(p-r)\times r} & 0_{(p-r)\times (q-r)}\\
%    \vdots & & \vdots\\
%    a_{m1}I_r & 0_{r\times(q-r)} & \cdots & a_{mn}I_r & 0_{r\times(q-r)}\\
%    0_{(p-r)\times r} & 0_{(p-r)\times (q-r)} & \cdots & 0_{(p-r)\times r} & 0_{(p-r)\times (q-r)}\\
%    \end{pmatrix}
    &= \begin{pmatrix}
    a_{11}I_r & \bm{0} & \cdots & a_{1n}I_r & \bm{0}\\
    \bm{0} & \bm{0} & \cdots & \bm{0} & \bm{0}\\
    \vdots & \vdots & & \vdots & \vdots\\
    a_{m1}I_r & \bm{0} & \cdots & a_{mn}I_r & \bm{0}\\
    \bm{0} & \bm{0} & \cdots & \bm{0} & \bm{0}\\
    \end{pmatrix}\\
    &= \begin{pmatrix}
    A & & & & & &\\
    & A & & & & &\\
    & & \ddots & & & &\\
    & & & A & & & &\\
    & & & & 0 & & \\
    & & & & & \ddots &\\
    & & & & & & 0\\
    \end{pmatrix}
\end{align*}
Let $R_2$ and $Q_2$ be products of elementary transformations such that :
\begin{align*}
    R_2AQ_2 &=\begin{bmatrix}
    I_r & 0_{r\times(n-r)}\\
    0_{(m-r)\times r} & 0_{(m-r)\times (n-r)}
    \end{bmatrix}
\end{align*}
\begin{align*}
    \begin{pmatrix}
    R_2 & & & & & \\
    & \ddots & & & & \\
    & & R_2 & & &\\
    & & & 0 & &\\
    & & & & \ddots &\\
    & & & & & 0
    \end{pmatrix}
    \begin{pmatrix}
    A & & & & & &\\
    & \ddots & & & &\\
    & & A & & & &\\
    & & & 0 & & \\
    & & & & \ddots &\\
    & & & & & 0\\
    \end{pmatrix}
    \begin{pmatrix}
    Q_2 & & & & & \\
    & \ddots & & & & \\
    & & Q_2 & & &\\
    & & & 0 & &\\
    & & & & \ddots &\\
    & & & & & 0
    \end{pmatrix}
\end{align*}
\end{proof}

% TODO other questions
\end{document}
