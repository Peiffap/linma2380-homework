%%%%%%%%%%%%%%%%%%%%%%%%%%%%%%%%%%%%%%%%%
% Jacobs Landscape Poster
% LaTeX Template
% Version 1.0 (29/03/13)
%
% Created by:
% Computational Physics and Biophysics Group, Jacobs University
% https://teamwork.jacobs-university.de:8443/confluence/display/CoPandBiG/LaTeX+Poster
% 
% Further modified by:
% Nathaniel Johnston (nathaniel@njohnston.ca)
%
% This template has been downloaded from:
% http://www.LaTeXTemplates.com
%
% 
% Masaryk University presentation themes were downloaded from:
% https://www.overleaf.com/gallery/tagged/muni
%
% and ported into Jacobs Landscape Poster by:
% Jumaidil Awal (ideal1st.here@googlemail.com)
% 
% Jacobs Landscape Poster License:
% CC BY-NC-SA 3.0 (http://creativecommons.org/licenses/by-nc-sa/3.0/)
%
% Masaryk University's fibeamer theme license:
% Copyright 2015  Vít Novotný <witiko@mail.muni.cz>
% Faculty of Informatics, Masaryk University (Brno, Czech Republic)
% under Latex Project Public License
%
%%%%%%%%%%%%%%%%%%%%%%%%%%%%%%%%%%%%%%%%%

%----------------------------------------------------------------------------------------
%	PACKAGES AND OTHER DOCUMENT CONFIGURATIONS
%----------------------------------------------------------------------------------------

\documentclass[final]{beamer}

\usepackage[scale=0.849]{beamerposter} % Use the beamerposter package for laying out the poster

%\usetheme{confposter} % Use the confposter theme supplied with this template
\usetheme[faculty=chemo]{fibeamer} % Uncomment to use Masaryk University's fibeamer theme instead.

%\setbeamercolor{block title}{fg=ngreen,bg=white} % Colors of the block titles
%\setbeamercolor{block body}{fg=black,bg=white} % Colors of the body of blocks
%\setbeamercolor{block alerted title}{fg=white,bg=ngreen} % Colors of the highlighted block titles
%\setbeamercolor{block alerted body}{fg=black,bg=dblue!10} % Colors of the body of highlighted blocks
% Many more colors are available for use in beamerthemeconfposter.sty

%-----------------------------------------------------------
% Define the column widths and overall poster size
% To set effective sepwid, onecolwid and twocolwid values, first choose how many columns you want and how much separation you want between columns
% In this template, the separation width chosen is 0.024 of the paper width and a 4-column layout
% onecolwid should therefore be (1-(# of columns+1)*sepwid)/# of columns e.g. (1-(4+1)*0.024)/4 = 0.22
% Set twocolwid to be (2*onecolwid)+sepwid = 0.464
% Set threecolwid to be (3*onecolwid)+2*sepwid = 0.708

\newlength{\sepwid}
\newlength{\onecolwid}
\newlength{\twocolwid}
\newlength{\threecolwid}
\setlength{\paperwidth}{46.8in} % A0 width: 46.8in
\setlength{\paperheight}{33.1in} % A0 height: 33.1in
\setlength{\sepwid}{0.007\paperwidth} % Separation width (white space) between columns
\setlength{\onecolwid}{0.21\paperwidth} % Width of one column
\setlength{\twocolwid}{0.451\paperwidth} % Width of two columns
\setlength{\threecolwid}{0.7\paperwidth} % Width of three columns
%\setlength{\topmargin}{-0.5in} % Reduce the top margin size
%-----------------------------------------------------------

% Theorems
\newtheorem{thm}{Theorem}[section]
%\newtheorem{proof}[thm]{Proof}
%\newtheorem{lemma}[thm]{Lemma}
\newtheorem{rem}[thm]{Remark}
\newtheorem{cor}[thm]{Corollary}
\newtheorem{ex}[thm]{Example}
\newtheorem{assu}[thm]{Assumption}
\newtheorem{alg}[thm]{Algorithm}
\newtheorem{defn}[thm]{Definition}

\usepackage{caption}
\captionsetup[figure]{labelformat=simple}


\usepackage{graphicx}  % Required for including images

\usepackage{booktabs} % Top and bottom rules for tables

%----------------------------------------------------------------------------------------
%	TITLE SECTION 
%----------------------------------------------------------------------------------------

\title{Nonnegative Matrix Factorization} % Poster title

\author{Group 2} % Author(s)

\institute{Ecole Polytechnique de Louvain} % Institution(s)

%----------------------------------------------------------------------------------------

\begin{document}
\addtobeamertemplate{block end}{}{\vspace*{2ex}} % White space under blocks
\addtobeamertemplate{block example end}{}{\vspace*{2ex}} % White space under example blocks
\addtobeamertemplate{block alerted end}{}{\vspace*{2ex}} % White space under highlighted (alert) blocks

\setlength{\belowcaptionskip}{2ex} % White space under figures
\setlength\belowdisplayshortskip{2ex} % White space under equations
%\begin{darkframes} % Uncomment for dark theme, don't forget to \end{darkframes}
\begin{frame} % The whole poster is enclosed in one beamer frame

%==========================Begin Head===============================

  \begin{columns}
   \begin{column}{\linewidth}
    \vskip1cm
    \centering
    \usebeamercolor{title in headline}{\color{fg}\Huge{\textbf{\inserttitle}}\\[0.3ex]}
    \usebeamercolor{author in headline}{\color{fg}\large{\insertauthor}\\[0.3ex]}
    %\usebeamercolor{institute in headline}{\color{fg}\large{\insertinstitute}\\[1ex]}
   \end{column}
   %\vspace{0.5cm}
  \end{columns}
 %\vspace{1cm}

%==========================End Head===============================

\begin{columns}[t] % The whole poster consists of three major columns, the second of which is split into two columns twice - the [t] option aligns each column's content to the top
\begin{column}{\sepwid}\end{column} % Empty spacer column
\begin{column}{\threecolwid}
\begin{columns}[t]

\begin{column}{\onecolwid} % The first column

%----------------------------------------------------------------------------------------
%	Introduction and Motivation
%----------------------------------------------------------------------------------------

\begin{exampleblock}{In a few words}
NMF is a powerful tool for the analysis of \textbf{high-dimensional} data as it automatically extracts \textbf{sparse} and \textbf{meaningful} features from a set of \textbf{nonnegative} vectors.
\end{exampleblock}

%----------------------------------------------------------------------------------------
%	Nonnegative Matrix Factorization: definition and properties (sparse, meaningful features)
%----------------------------------------------------------------------------------------

\begin{exampleblock}{Nonnegative Matrix Factorization : definition and properties}

\end{exampleblock}

\end{column} % End of the first column

\begin{column}{\sepwid}\end{column} % Empty spacer column

\begin{column}{\twocolwid} % Begin a column which is two columns wide (column 2)

%----------------------------------------------------------------------------------------
%	Applications
%----------------------------------------------------------------------------------------

\begin{exampleblock}{Applications}
\textbf{Image processing}\\
Data matrix:\\
The data matrix $X\in\real^{p\times n}_+$ carries information about $n$ image faces,
\begin{itemize}
    \item Each image has $q$ pixels
    \item Each column of the data matrix $X$ represents an image of a face
    \item The $(i,j)$th entry of $X$ represents the gray-level of the $i$th pixel in the $j$th face
\end{itemize}
NMF factorization:\\
\begin{itemize}
    \item Each column of the matrix $W$ represents a facial feature.
    \item The $(i,j)$th entry of $H$ represents the importance of the $i$-th feature in the $j$-th face
\end{itemize}
\end{exampleblock}

\begin{exampleblock}{Algorithms and difficulties}
TO DO
\end{exampleblock}

\end{column} % End of the second column

\end{columns}

\begin{alertblock}{Conclusion}

TO DO

\end{alertblock}


\end{column}
\begin{column}{\sepwid}\end{column} % Empty spacer column
\begin{column}{\onecolwid} % The third column

%----------------------------------------------------------------------------------------
%	Links to other problems
%----------------------------------------------------------------------------------------

\begin{exampleblock}{Links to other problems}

TO DO
\end{exampleblock}

%----------------------------------------------------------------------------------------
%	References
%----------------------------------------------------------------------------------------
\begin{alertblock}{References}

\begin{thebibliography}{99}
\end{thebibliography}

\end{alertblock}
\end{column} % End of the third column
\begin{column}{\sepwid}\end{column} % Empty spacer column
\end{columns} % End of all the columns in the poster
\end{frame} % End of the enclosing frame
%\end{darkframes} % Uncomment for dark theme
\end{document}
