\documentclass[11pt]{article}
\usepackage[a4paper,left=1.5cm,right=1.5cm,top=1.5cm,bottom=1.5cm]{geometry}
\usepackage{fancyhdr}
\usepackage{mleftright}
\usepackage{verbatim}
\renewcommand{\headrulewidth}{1pt}
\fancyhead[C]{\textsc{[LINMA2380] --- Homework 2}}
\fancyhead[L]{2 November 2020}
\fancyhead[R]{Group 02}

\usepackage{tikz}
\usepackage{pgfplots}
\usepackage[T1]{fontenc}
\usepackage[utf8]{inputenc}
\usepackage[english]{babel}
\usepackage{graphicx}
\usepackage{subcaption}
\usepackage{csquotes}
\usepackage{mathtools,amssymb,amsthm}
\usepackage[binary-units=true,separate-uncertainty = true,multi-part-units=single]{siunitx}
\usepackage{float}
\usepackage[linktoc=all]{hyperref}
\hypersetup{breaklinks=true}
\graphicspath{{img/}}
\usepackage{caption}
\usepackage{textcomp}
\usepackage{array}
\usepackage{color}
\usepackage{tabularx,booktabs}
\usepackage{titlesec}
\usepackage{wrapfig}
\pagestyle{fancy}
\usepackage{mathrsfs}
\usepackage{bm}
\DeclarePairedDelimiterX{\norm}[1]{\lVert}{\rVert}{#1}

\newcommand{\imag}{\mathrm{i}\mkern1mu} % Imaginary unit
\newcommand{\abs}[1]{\left\lvert#1\right\lvert}
\usepackage{listings}
\lstset{
    language=Python,
    numbers=left,
    numberstyle=\tiny\color{gray},
    basicstyle=\rm\small\ttfamily,
    keywordstyle=\bfseries\color{dkred},
    frame=single,
    commentstyle=\color{gray}=small,
    stringstyle=\color{dkgreen},
    %backgroundcolor=\color{gray!10},
    %tabsize=8, % Thank you Papa Torvalds
    %rulecolor=\color{black!30},
    %title=\lstname,
    breaklines=true,
    framextopmargin=2pt,
    framexbottommargin=2pt,
    extendedchars=true,
    inputencoding=utf8,
}

\DeclareMathOperator{\rank}{rank}
\DeclareMathOperator{\vect}{vec}
\DeclareMathOperator{\newdiff}{d} % use \dif instead
\newcommand{\dif}{\newdiff\!}
\newcommand{\e}{\mathrm{e}}

\newcommand{\field}{\mathbb{F}} % field
\newcommand{\real}{\mathbb{R}} % real numbers
\newcommand{\complex}{\mathbb{C}} % complex numbers

\newcommand{\snorm}[1]{\norm{#1}_2} % spectral norm
\newcommand{\fnorm}[1]{\norm{#1}_F} % frobenius norm

\setcounter{MaxMatrixCols}{15}

\newcommand\undermat[2]{% http://tex.stackexchange.com/a/102468/5764
    \makebox[0pt][l]{$\smash{\underbrace{\phantom{%
                    \begin{matrix}#2\end{matrix}}}_{\text{$#1$}}}$}#2}

\begin{document}
\section*{Exercise A: Least square problems}
\subsection*{A1}
\subsection*{A2}
Suppose the QR decomposition of $A$ is given by $\tiny{Q \begin{pmatrix}R\\0\end{pmatrix}}$, where $Q \in \real^{m\times m}$ is unitary and $R \in \real^{n\times n}$ is upper triangular. We will express the solution of 
\begin{equation}\label{eqA2}
    A^T Ax=A^T b
\end{equation} in terms of the QR decomposition of A.\\
Let $\tiny{R_f=\begin{pmatrix}R\\0\end{pmatrix}}$. We can rewrite equation \eqref{eqA2} as:
\begin{align*}
    (QR_f)^{T}QR_fx &= (QR_f)^Tb\\
    R_f^{T}Q^{T}QR_fx &= R_f^{T}Q^Tb
\end{align*}
As Q is unitary and hence $Q^TQ=I$, we have:
\begin{align*}
    \begin{pmatrix}
    R^T & 0
    \end{pmatrix}
    \begin{pmatrix}
    R\\ 0
    \end{pmatrix}x
    &=\begin{pmatrix}
    R^T & 0
    \end{pmatrix}
    Q^Tb
\end{align*}
If we call $\hat{Q}$ the matrix consisting of the first n columns of Q, equation \eqref{eqA2} becomes:
\begin{align*}
    R^TRx &= R^T\hat{Q}^Tb
\end{align*}
From theorem 2.8 of the course notes, we know that every matrix $A\in\complex^{m\times n}$ of full column-rank admits a factorization $A=Q_1R_1$ where $Q_1\in\complex^{m\times n}$ is an isometry and $R_1\in\complex^{n\times n}$ is an upper triangular matrix with positive diagonal. The matrix $Q_1$ corresponds to $\hat{Q}$ and $R_1$ simply to $R$. Hence we deduce that $R$ is invertible. This allows us to premultiply both sides of the equation by the inverse of $R^T$:
\begin{align*}
    R^{-T}R^TRx &= R^{-T}R^T\hat{Q}^Tb\\
    Rx &= \hat{Q}^Tb
\end{align*}
The solution of equation \eqref{eqA2} is therefore $x=R^{-1}\hat{Q}^Tb$ and the computation of the solution is reduced to the resolution of a single triangular system of linear equations (which can be solved efficiently using backward substitution).

\subsection*{A3}

\section*{Exercise B: Low-rank approximation}
\subsection*{B1}
For every matrix \(A \in \real^{m \times n}\), there exist unitary transformations \(U \in \real^{m \times m}\) and \(V \in \real^{n \times n}\) such that
\[
A = U \Sigma V^*, \quad \textnormal{where}\quad \Sigma = \left[\begin{array}{ccc|c}
\sigma_1 & & 0 & \\
& \ddots & & 0_{r \times (n-r)}\\
0 && \sigma_r & \\
\hline
& 0_{(m-r) \times r} & & 0_{(m-r) \times (n-r)}
\end{array}\right],
\]
with real positive singular values \(\sigma_1 \geqslant \dots \geqslant \sigma_r > 0\).

These singular values are unique: the intuition to see this is that the singular value decomposition is computed inductively, and that the unitary matrices preserve the norm.
By taking the property that \(\snorm{X} = \sigma_1\), this means that at every step of the decomposition, no matter what unitary transformations are chosen, the norm (and thus the maximal singular value of the submatrix we are working on) is the same.

Next, we show that the rank of a matrix is equal to its number of nonzero singular values.
\begin{proof}
	We know that the rank of a diagonal matrix is equal to the number of its nonzero entries.
	We also note that in the decomposition \(A = U \Sigma V\), \(U\) and \(V\) are of full rank.
	Therefore, \(\rank(A) = \rank(\Sigma) = r\).
\end{proof}

\subsection*{B2}
Let \(x \in \real^{m \times n}\) be such that \(\abs{X_{ij}} \leqslant \varepsilon\) for all \(i \in \{1, \dots, m\}\) and \(j \in \{1, \dots, n\}\).
Let \(\snorm{X}\) be the \(2\)-norm of \(X\) and let \(\fnorm{X}\) be its Frobenius norm.
We show that \(\snorm{X} \leqslant \fnorm{X} \leqslant \sqrt{mn} \varepsilon\).
\begin{proof}
	First, we show the first inequality.
	We know from the lecture notes that
	\begin{align*}
	\snorm{X} &= \sigma_{\textnormal{max}},\\
	\fnorm{X} &= \left[\sum_i \sigma_i\right]^{1/2},
	\end{align*}
	where \(\sigma_i\) are the singular values of \(X\).
	From this, it is immediately clear that \(\snorm{X} \leqslant \fnorm{X}\).
	
	Next, we use an equivalent form of the Frobenius norm to show the second inequality:
	\[
	\fnorm{X} = \left[\sum_{i, j} \abs{X_{ij}}^2 \right]^{1/2}.
	\]
	Knowing that \(\abs{X_{ij}} \leqslant \varepsilon\), it is immediate that \(\fnorm{X} \leqslant \left[\sum_{i, j} \varepsilon^2\right]^{1/2} = \left[mn \varepsilon^2\right]^{1/2} = \sqrt{mn} \varepsilon\).
	This concludes the proof.
\end{proof}

We also give an example where these bounds are tight.
Indeed, consider the matrix \(X = I_1 \in \real^{1 \times 1}\).
Clearly, we have \(\abs{X_{ij}} \leqslant \varepsilon = 1\) for all \(i, j\) (only one value is possible for each).
We know that the only singular value of this matrix is \(1\), and hence
\[
\snorm{X} = \fnorm{X} = \sqrt{1 \cdot 1} \varepsilon = 1.
\]
\subsection*{B3}

\section*{Exercise C: Low-rank approximation}
\subsection*{Discussion}
\subsection*{Bonus question}


\end{document}
