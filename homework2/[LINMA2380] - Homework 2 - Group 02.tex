\documentclass[11pt]{article}
\usepackage[a4paper,left=1.5cm,right=1.5cm,top=1.5cm,bottom=1.5cm]{geometry}
\usepackage{fancyhdr}
\usepackage{mleftright}
\usepackage{verbatim}
\renewcommand{\headrulewidth}{1pt}
\fancyhead[C]{\textsc{[LINMA2380] --- Homework 2}}
\fancyhead[L]{2 November 2020}
\fancyhead[R]{Group 02}

\usepackage{tikz}
\usepackage{pgfplots}
\usepackage[T1]{fontenc}
\usepackage[utf8]{inputenc}
\usepackage[english]{babel}
\usepackage{graphicx}
\usepackage{subcaption}
\usepackage{csquotes}
\usepackage{mathtools,amssymb,amsthm}
\usepackage[binary-units=true,separate-uncertainty = true,multi-part-units=single]{siunitx}
\usepackage{float}
\usepackage[linktoc=all]{hyperref}
\hypersetup{breaklinks=true}
\graphicspath{{img/}}
\usepackage{caption}
\usepackage{textcomp}
\usepackage{array}
\usepackage{color}
\usepackage{tabularx,booktabs}
\usepackage{titlesec}
\usepackage{wrapfig}
\pagestyle{fancy}
\usepackage{mathrsfs}
\usepackage{bm}
\DeclarePairedDelimiterX{\norm}[1]{\lVert}{\rVert}{#1}

\newcommand{\imag}{\mathrm{i}\mkern1mu} % Imaginary unit
\newcommand{\abs}[1]{\left\lvert#1\right\lvert}
\usepackage{listings}
\lstset{
    language=Python,
    numbers=left,
    numberstyle=\tiny\color{gray},
    basicstyle=\rm\small\ttfamily,
    keywordstyle=\bfseries\color{dkred},
    frame=single,
    commentstyle=\color{gray}=small,
    stringstyle=\color{dkgreen},
    %backgroundcolor=\color{gray!10},
    %tabsize=8, % Thank you Papa Torvalds
    %rulecolor=\color{black!30},
    %title=\lstname,
    breaklines=true,
    framextopmargin=2pt,
    framexbottommargin=2pt,
    extendedchars=true,
    inputencoding=utf8,
}

\DeclareMathOperator{\rank}{rank}
\DeclareMathOperator{\vect}{vec}
\DeclareMathOperator{\newdiff}{d} % use \dif instead
\newcommand{\dif}{\newdiff\!}
\newcommand{\e}{\mathrm{e}}

\newcommand{\field}{\mathbb{F}} % field
\newcommand{\real}{\mathbb{R}} % real numbers
\newcommand{\complex}{\mathbb{C}} % complex numbers

\newcommand{\snorm}[1]{\norm{#1}_2} % spectral norm
\newcommand{\fnorm}[1]{\norm{#1}_F} % frobenius norm

\setcounter{MaxMatrixCols}{15}

\newcommand\undermat[2]{% http://tex.stackexchange.com/a/102468/5764
    \makebox[0pt][l]{$\smash{\underbrace{\phantom{%
                    \begin{matrix}#2\end{matrix}}}_{\text{$#1$}}}$}#2}

\begin{document}
\section*{Exercise A: Least square problems}
\subsection*{A1}
\subsection*{A2}
\subsection*{A3}

\section*{Exercise B: Low-rank approximation}
\subsection*{B1}
\subsection*{B2}
Let \(x \in \real^{m \times n}\) be such that \(\abs{X_{ij}} \leqslant \varepsilon\) for all \(i \in \{1, \dots, m\}\) and \(j \in \{1, \dots, n\}\).
Let \(\snorm{X}\) be the \(2\)-norm of \(X\) and let \(\fnorm{X}\) be its Frobenius norm.
We show that \(\snorm{X} \leqslant \fnorm{X} \leqslant \sqrt{mn} \varepsilon\).
\begin{proof}
	First, we show the first inequality.
	We know from the lecture notes that
	\begin{align*}
	\snorm{X} &= \sigma_{\textnormal{max}},\\
	\fnorm{X} &= \left[\sum_i \sigma_i\right]^{1/2},
	\end{align*}
	where \(\sigma_i\) are the singular values of \(X\).
	From this, it is immediately clear that \(\snorm{X} \leqslant \fnorm{X}\).
	
	Next, we use an equivalent form of the Frobenius norm to show the second inequality:
	\[
	\fnorm{X} = \left[\sum_{i, j} \abs{X_{ij}}^2 \right]^{1/2}.
	\]
	Knowing that \(\abs{X_{ij}} \leqslant \varepsilon\), it is immediate that \(\fnorm{X} \leqslant \left[\sum_{i, j} \varepsilon^2\right]^{1/2} = \left[mn \varepsilon^2\right]^{1/2} = \sqrt{mn} \varepsilon\).
	This concludes the proof.
\end{proof}

We also give an example where these bounds are tight.
Indeed, consider the matrix \(X = I_1 \in \real^{1 \times 1}\).
Clearly, we have \(\abs{X_{ij}} \leqslant \varepsilon = 1\) for all \(i, j\) (only one value is possible for each).
We know that the only singular value of this matrix is \(1\), and hence
\[
\snorm{X} = \fnorm{X} = \sqrt{1 \cdot 1} \varepsilon = 1.
\]
\subsection*{B3}

\section*{Exercise C: Low-rank approximation}
\subsection*{Discussion}
\subsection*{Bonus question}


\end{document}
